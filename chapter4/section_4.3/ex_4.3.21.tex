% === Exercise 4.3.21 ===
\begin{Exercise}
\begin{proof}
For $n=2$, we know $M = \begin{pmatrix}
A & B \\
O_1 & C
\end{pmatrix}$ where $A,B,C,O_1\in M_{1\times 1}(F)$. Then $\det(M) = \det(A)\det(C)$ by definition. Suppose for $n-1$, the statement holds; then for $n$, we know the cofactor expansion of $M$ along the first column gives
$$
\det(M) = \sum_{i=1}^{n}(-1)^{i+1} M_{i 1} \det(\tilde{M}_{i 1}).
$$
Assume $A$ is a $k\times k$ matrix where $k<n$. Since $M_{i 1} = 0$ for $i>k$. The formula leads to
$$
\det(M) = \sum_{i=1}^{k}(-1)^{i+1} M_{i 1} \det(\tilde{M}_{i 1}).
$$
Suppose
$
\tilde{M}_{i 1} = \begin{pmatrix}
\tilde{A}_{i 1} & \tilde{B} \\
O_{n-k-1} & C
\end{pmatrix}
$
where $\tilde{B}$ is obtained from deleting the $i$-th row of $B$. Since $\tilde{M}_{i 1}$ is a $(n-1)\times(n-1)$ matrix, by supposition of induction, we have $\det(\tilde{M}_{i 1}) = \det(\tilde{A}_{i 1})\det(C)$. Hence
$$
\det(M) = \sum_{i=1}^{k}(-1)^{i+1} M_{i 1} \det(\tilde{A}_{i 1})\det(C).
$$
Notice that $M_{i 1} = A_{i 1}$ for each $i=1,2,\cdots,k$, then it follows that
$$
\det(M) = \left(\sum_{i=1}^{k}(-1)^{i+1} A_{i 1} \det(\tilde{A}_{i 1}) \right)\det(C)
= \det(A)\det(C).
$$
This means the statement also holds for $n$.

By induction, we conclude the statement holds for $n\geq 2$.
\end{proof}
\end{Exercise}