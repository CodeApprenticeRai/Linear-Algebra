% === Exercise 4.1.11 ===
\begin{Exercise}
	\begin{proof}
		First, we need to know the values of $\delta\begin{pmatrix}
		1 & 0 \\
		1 & 0
		\end{pmatrix}, \delta\begin{pmatrix}
		1 & 0 \\
		0 & 1
		\end{pmatrix}, \delta\begin{pmatrix}
		0 & 1 \\
		1 & 0
		\end{pmatrix}, \delta\begin{pmatrix}
		0 & 1 \\
		0 & 1
		\end{pmatrix}$.
		
		By property (ii), we know 
		$$
		\delta\begin{pmatrix}
		1 & 0 \\
		1 & 0
		\end{pmatrix} = \delta\begin{pmatrix}
		0 & 1 \\
		0 & 1
		\end{pmatrix} = 0;
		$$
		also by property (iii), we know $\delta\begin{pmatrix}
		1 & 0 \\
		0 & 1
		\end{pmatrix} = 1$. Since
		\begin{align*}
		0 = \delta\begin{pmatrix}
		1 & 1 \\
		1 & 1
		\end{pmatrix} 
		&= \delta\begin{pmatrix}
		1 & 0 \\
		1 & 1
		\end{pmatrix} + \delta\begin{pmatrix}
		0 & 1 \\
		1 & 1
		\end{pmatrix} \\
		&= \left( \delta\begin{pmatrix}
		1 & 0 \\
		1 & 0
		\end{pmatrix} + \delta\begin{pmatrix}
		1 & 0 \\
		0 & 1
		\end{pmatrix} \right) + \left( \delta\begin{pmatrix}
		0 & 1 \\
		1 & 0
		\end{pmatrix} + \delta\begin{pmatrix}
		0 & 1 \\
		0 & 1
		\end{pmatrix} \right) \\
		&= ( 0 + 1 ) + \left(\delta\begin{pmatrix}
		0 & 1 \\
		1 & 0
		\end{pmatrix} + 0 \right),
		\end{align*}
		this implies $\delta\begin{pmatrix}
		0 & 1 \\
		1 & 0
		\end{pmatrix} = -1$.
		
		Let $A = \begin{pmatrix}
		a & b \\
		c & d
		\end{pmatrix}\in M_{2\times 2}(F)$. Now consider
		\begin{align*}
		\delta(A)
		&= \delta\begin{pmatrix}
		a & b \\
		c & d
		\end{pmatrix} \\
		&= a\delta\begin{pmatrix}
		1 & 0 \\
		c & d
		\end{pmatrix} + b\delta\begin{pmatrix}
		0 & 1 \\
		c & d
		\end{pmatrix} \\
		&= a\left(c\delta\begin{pmatrix}
		1 & 0 \\
		1 & 0
		\end{pmatrix} + d\delta\begin{pmatrix}
		1 & 0 \\
		0 & 1
		\end{pmatrix}\right) + b\left( c\delta\begin{pmatrix}
		0 & 1 \\
		1 & 0
		\end{pmatrix} + d\delta\begin{pmatrix}
		0 & 1 \\
		0 & 1
		\end{pmatrix}\right) \\
		&= a d - b c \\
		&= \det(A).
		\end{align*}
		Hence we complete the proof.
	\end{proof}
\end{Exercise}