% === Exercise 3.2.14 ===
\begin{Exercise}
\begin{enumerate}[(a)]
\item
\begin{proof}
Let arbitrary $w\in R(T+U)$, then there exists $v\in V$ such that $w=(T+U)(v) = T(v)+U(v)$. Since $T(v) \subseteq R(T)$ and $U(v) \subseteq R(U)$, we have $T(v)+U(v) \subseteq R(T)+R(U)$. Because $w$ was arbitrary, we conclude $R(T+U)\subseteq R(T)+R(U)$.
\end{proof}

\item
\begin{proof}
Since $dim(W)$ is finite, by Exercise 1.6.29(a), we have
\begin{align*}
rank(T+U)
&\leq dim(R(T)+R(U)) \\
&= dim(R(T)) + dim(R(U)) - dim(R(T)\cap R(U)) \\
&\leq dim(R(T)) + dim(R(U)) \\
&= rank(T)+rank(U).
\end{align*}
\end{proof}

\item
\begin{proof}
Form part (b), we consider $rank(A+B) = rank(L_{A+B}) = rank(L_A+L_B) \leq rank(L_A) + rank(L_B) = rank(A) + rank(B)$.
\end{proof}

\end{enumerate}
\end{Exercise}