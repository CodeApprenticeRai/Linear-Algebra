% === Exercise 3.3.8 ===
\begin{Exercise}
\begin{enumerate}[(a)]
\item[(a)]
\begin{answer}
No.
\end{answer}
\begin{solution}
This means whether $\begin{cases}
a+b = 1 \\
b-2c = 3 \\
a+2c = -2
\end{cases}$ has a solution or not. After computing, we have
$\begin{pmatrix} \left.\begin{matrix}
1 & 1 & 0 \\
0 & 1 & -2 \\
0 & 0 & 0
\end{matrix} \right| \begin{matrix}
1 \\
3 \\
4
\end{matrix} \end{pmatrix}$, it implies the systems of linear equations has no solution. Hence $v\notin R(T)$.
\end{solution}

\item[(b)]
\begin{answer}
Yes.
\end{answer}
\begin{solution}
This means whether $\begin{cases}
a+b = 2\\
b-2c = 1 \\
a+2c = 1
\end{cases}$ has a solution or not. After computing, we have $\begin{pmatrix} \left.\begin{matrix}
1 & 1 & 0 \\
0 & 1 & -2 \\
0 & 0 & 0
\end{matrix} \right| \begin{matrix}
2 \\
1 \\
0
\end{matrix} \end{pmatrix}$, it implies $T(-1,3,1) = (2,1,1) = v$. Hence $v\in R(T)$.
\end{solution}

\end{enumerate}
\end{Exercise}