% === Exercise 3.3.1 ===
\begin{Exercise}
	\begin{enumerate}[(a)]
		\item[(a)]
		\begin{answer}
			False.
		\end{answer}
		\begin{solution}
			Consider $\begin{cases}
			x = 0 \\
			x = 1
			\end{cases}$. Then it has no solution.
		\end{solution}
		
		\item[(b)]
		\begin{answer}
			False.
		\end{answer}
		\begin{solution}
			Consider $\begin{cases}
			x_1 + x_2 = 2 \\
			2x_1 + 2x_2 = 4
			\end{cases}$. Then it has infinitely many solutions.
		\end{solution}
		
		\item[(c)]
		\begin{answer}
			True.
		\end{answer}
		\begin{solution}
			The solutions at least contain zero solution trivially.
		\end{solution}
		
		\item[(d)]
		\begin{answer}
			False.
		\end{answer}
		\begin{solution}
			Consider $\begin{cases}
			x_1 + x_2 = 2 \\
			2x_1 + 2x_2 = 4
			\end{cases}$. Then it has infinitely many solutions.
		\end{solution}
		
		\item[(e)]
		\begin{answer}
			False.
		\end{answer}
		\begin{solution}
			Consider $\begin{cases}
			x_1 + x_2 = 0 \\
			x_1 + x_2 = 1
			\end{cases}$. Then it has no solution.
		\end{solution}
		
		\item[(f)]
		\begin{answer}
			False.
		\end{answer}
		\begin{solution}
			Pick $0x = 0$. Then it has infinitely many solutions. However, $0x = 1$ has no solution.
		\end{solution}
		
		\item[(g)]
		\begin{answer}
			True.
		\end{answer}
		\begin{solution}
			Let $A$ be a $n\times n$ matrix. Then we know $Ax = 0$. It follows that $x = A^{-1} 0 = 0$. So the solution must be zero.
		\end{solution}
		
		\item[(h)]
		\begin{answer}
			False.
		\end{answer}
		\begin{solution}
			Since $x = 1$ has the solution set $\{1\}$, it has no zero vector in $F^1$, then it violate the definition of subspace.
		\end{solution}
		
	\end{enumerate}
\end{Exercise}