% === Exercise 7.1.1 ===
\begin{Exercise}
\begin{enumerate}[(a)]
\item[(a)]
\begin{answer}
True.
\end{answer}
\begin{solution}
It follows from definition.
\end{solution}

\item[(b)]
\begin{answer}
False.
\end{answer}
\begin{solution}
Let $x$ be a generalized eigenvector of $T$, then by definition, there exists $p$ is the smallest integer such that $(T-\lambda I)^p(x) = 0$. However $y = (T-\lambda I)^{p-1}\neq 0$ is an eigenvector corresponding to $\lambda$. This means $\lambda$ is an eigenvalue.
\end{solution}

\item[(c)]
\begin{answer}
False.
\end{answer}
\begin{solution}
It also need to satisfy the characteristic polynomial can split. It follows from the Corollary 1 of Theorem 7.7.
\end{solution}

\item[(d)]
\begin{answer}
True.
\end{answer}
\begin{solution}
It follows from the Corollary of Theorem 7.6.
\end{solution}

\item[(e)]
\begin{answer}
False.
\end{answer}
\begin{solution}
Consider $A = \begin{pmatrix}
2 & -1 & 0 & 1 \\
0 & 3 & -1 & 0 \\
0 & 1 & 1 & 0 \\
0 & -1 & 0 & 3
\end{pmatrix}$. Corresponding to the eigenvalue $2$, we have two cycle of generalized eigenvectors of $A$.
\end{solution}

\item[(f)]
\begin{answer}
False.
\end{answer}
\begin{solution}
Consider $T = \begin{pmatrix}
1 & 1 \\
0 & 1
\end{pmatrix}$. Observe the eigenvalue is $1$. Then $K_1 = \mathbb{R}^2$. We pick $\beta = \{(1,-1),(1,1)\}$ to know that $[T]_{\beta} = \begin{pmatrix}
1 & 2 \\
-1 & 0
\end{pmatrix}$ is not a Jordan canonical form.
\end{solution}

\item[(g)]
\begin{answer}
True.
\end{answer}
\begin{solution}
We just pick $\beta$ be the standard basis. Then $[L_\mathcal{J}]_{\beta} = \mathcal{J}$.
\end{solution}

\item[(h)]
\begin{answer}
True.
\end{answer}
\begin{solution}
It follows from Theorem 7.2(b).
\end{solution}

\end{enumerate}
\end{Exercise}