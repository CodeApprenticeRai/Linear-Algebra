% === Exercise 7.1.13 ===
\begin{Exercise}
	\begin{proof}
		Let $V = K_{\lambda_1}\oplus K_{\lambda_2} \oplus \cdots \oplus K_{\lambda_k}$ by Theorem 7.8. We pick a basis $\beta_i$ which is a union of cycles of generalized eigenvectors for $T_{K_{\lambda_i}}$ for each $i=1,2,\cdots,k$. This means $\mathcal{J}_i = [T_{K_{\lambda_i}}]_{\beta_i}$ is a Jordan canonical form for $T_{K_{\lambda_i}}$. Put $\beta = \cup_{i=1}^{k}\beta_i$. Then $\beta$ is a basis for $V$ where $\beta$ is a union of cycles of generalized eigenvectors in $T$. Hence $\mathcal{J} = [T]_{\beta}$ is a Jordan canonical form for $T$. Moreover, by Theorem 5.25, $\mathcal{J} = \mathcal{J}_1\oplus\mathcal{J}_2\oplus\cdots\oplus\mathcal{J}_k$ as promised.
	\end{proof}
\end{Exercise}