% === Exercise 7.2.1 ===
\begin{Exercise}
\begin{enumerate}[(a)]
\item[(a)]
\begin{answer}
True.
\end{answer}
\begin{solution}
Since the Jordan form of a diagonal matrix is itself. And the Jordan canonical form is unique.
\end{solution}

\item[(b)]
\begin{answer}
True.
\end{answer}
\begin{solution}
Since $T$ is similar to $[T]_{\beta}$ for arbitrary $\beta$, then by Theorem 7.11, we know they have the same Jordan canonical form $\mathcal{J}$.
\end{solution}

\item[(c)]
\begin{answer}
False.
\end{answer}
\begin{solution}
Consider two matrices $\begin{pmatrix}
1 & 0 \\
0 & 1 
\end{pmatrix}$ and $\begin{pmatrix}
1 & 1 \\
0 & 1
\end{pmatrix}$, they have the same characteristic polynomial $(\lambda-1)^2$; however, they are not similar.
\end{solution}

\item[(d)]
\begin{answer}
True.
\end{answer}
\begin{solution}
It follows from Theorem 7.11.
\end{solution}

\item[(e)]
\begin{answer}
True.
\end{answer}
\begin{solution}
If we use a Jordan canonical basis $\beta$ to represent a matrix $T$, then we get the Jordan canonical form $\mathcal{J}$. Hence $T$ is similar to $\mathcal{J} = [T]_{\beta}$.
\end{solution}

\item[(f)]
\begin{answer}
False.
\end{answer}
\begin{solution}
Consider a particular case for which $t = 1$ and $n = 2$. Then we consider two matrices $\begin{pmatrix}
1 & 0 \\
0 & 1
\end{pmatrix}$ and $\begin{pmatrix}
1 & 1 \\
0 & 1
\end{pmatrix}$. They have different Jordan canonical forms.
\end{solution}

\item[(g)]
\begin{answer}
False.
\end{answer}
\begin{solution}
Consider the identity linear operator $T$, then $T = \begin{pmatrix}
1 & 0 \\
0 & 1
\end{pmatrix}$. 

We can pick $\{(0,1),(1,0)\}$ or $\{(1,1),(-1,1)\}$ to be a Jordan canonical basis for $T$.
\end{solution}

\item[(h)]
\begin{answer}
True.
\end{answer}
\begin{solution}
It follows from the Corollary of Theorem 7.10.
\end{solution}

\end{enumerate}
\end{Exercise}