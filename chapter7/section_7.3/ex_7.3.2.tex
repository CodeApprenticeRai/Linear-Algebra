% === Exercise 7.3.2 ===
\begin{Exercise}
	In spite of checking all possibilities, we can find Jordan canonical form in an alternative way by Exercise 7.3.13.
	\begin{enumerate}[(a)]
		\item[(a)]
		\begin{answer}
			$p(t) = (t-1)(t-3)$. 	
		\end{answer}
		\begin{solution}
			Find the Jordan canonical form $\begin{pmatrix}
			1 & 0 \\
			0 & 3
			\end{pmatrix}$.
			Hence we know the minimal polynomial is $(t-1)(t-3)$.
		\end{solution}
		
		\item[(b)]
		\begin{answer}
			$p(t) = (t-1)^2$. 	
		\end{answer}
		\begin{solution}	
			Find the Jordan canonical form $\begin{pmatrix}
			1 & 1 \\
			0 & 1
			\end{pmatrix}$.
			Hence we know the minimal polynomial is $(t-1)^2$.
		\end{solution}
		
		\item[(c)]
		\begin{answer}
			$p(t) = (t-1)^2(t-2)$. 	
		\end{answer}
		\begin{solution}
			Find the Jordan canonical form $\begin{pmatrix}
			1 & 1 & 0 \\
			0 & 1 & 0 \\
			0 & 0 & 2
			\end{pmatrix}$.
			Hence we know the minimal polynomial is $(t-1)^2(t-2)$.
		\end{solution}
		\item[(d)]
		\begin{answer}
			$p(t) = (t-2)^2$.
		\end{answer}
		\begin{solution}
			Let the matrix be $A$.
			We need to find the characteristic polynomial at first.
			Solve $\det(A-t I)= 0$ to obtain $-(t-2)^3$.
			
			Since the minimal polynomial divides the characteristic polynomial, we just check all possibilities one by one.
			Check $(t-2)$, then $A-2I = \begin{pmatrix}
			1 & 0 & 1 \\
			2 & 0 & 2 \\
			-1 & 0 & -1
			\end{pmatrix} \neq O$. Then check $(t-2)^2$, it follows that $(A-2I)^2 = O$. Hence the minimal polynomial is $(t-2)^2$
		\end{solution}
	\end{enumerate}
\end{Exercise} 