% === Exercise 7.3.3 ===
\begin{Exercise}
	Find $[T]_{\beta}$ and use a similar argument from previous exercise. Here we use the standard basis $\beta$.
	\begin{enumerate}[(a)]
		\item[(a)]
		\begin{answer}
			$p(t) = (t-\sqrt{2})(t+\sqrt{2})$. 	
		\end{answer}
		\begin{solution}
			Calculate $[T]_{\beta} = \begin{pmatrix}
			1 & 1 \\
			1 & -1
			\end{pmatrix}$.
			Then the Jordan canonical form is $\begin{pmatrix}
			-\sqrt{2} & 0 \\
			0 & \sqrt{2}
			\end{pmatrix}$.
			Hence the minimal polynomial is $(t-\sqrt{2})(t+\sqrt{2})$.
		\end{solution}
		
		\item[(b)]
		\begin{answer}
			$p(t) = (t-2)^3$. 	
		\end{answer}
		\begin{solution}	
			Calculate $[T]_{\beta} = \begin{pmatrix}
			2 & 1 & 0 \\
			0 & 2 & 2 \\
			0 & 0 & 2
			\end{pmatrix}$.
			Then the Jordan canonical form is $\begin{pmatrix}
			2 & 1 & 0 \\
			0 & 2 & 1 \\
			0 & 0 & 2
			\end{pmatrix}$.
			Hence the minimal polynomial is $(t-2)^3$.
		\end{solution}
		
		\item[(c)]
		\begin{answer}
			$p(t) = (t-2)^2$. 	
		\end{answer}
		\begin{solution}
			Calculate $[T]_{\beta} = \begin{pmatrix}
			2 & 1 & 0 \\
			0 & 2 & 0 \\
			0 & 0 & 2
			\end{pmatrix}$ which is the Jordan canonical form coincidentally. 
			Hence we know the minimal polynomial is $(t-2)^2$.
		\end{solution}
		\item[(d)]
		\begin{answer}
			$p(t) = (t-1)(t+1)$.
		\end{answer}
		\begin{solution}
			Following the hint, we have $(T-I)(T+I) = O$.
			Put $f(t) = (t-1)(t+1)$ with $f(T) = O$.
			By the Cayley-Hamilton Theorem, $f$ is the characteristic polynomial.
			Since $T-I \neq O$ and $T+I \neq O$, then the minimal polynomial must be $(t-1)(t+1)$.
		\end{solution}
	\end{enumerate}
\end{Exercise} 