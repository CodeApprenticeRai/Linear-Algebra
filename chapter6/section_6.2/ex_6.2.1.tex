% === Exercise 6.2.1 ===
\begin{Exercise}
\begin{enumerate}[(a)]
\item[(a)]
\begin{answer}
False.
\end{answer}
\begin{solution}
Sets of vectors must be linearly independent.
\end{solution}

\item[(b)]
\begin{answer}
True.
\end{answer}
\begin{solution}
It follows from Theorem 6.5.
\end{solution}

\item[(c)]
\begin{answer}
True.
\end{answer}
\begin{solution}
Let $V$ be an arbitrary inner product space and $W$ be a subspace of $V$. Then let $x,y\in W^{\perp}$ and $c$ be a scalar. For all $w\in W$,
$$
\langle x+y, w \rangle = \langle x,w \rangle + \langle y,w \rangle = 0 + 0 = 0;
$$
also
$$
\langle c x, w \rangle = c \langle x,w \rangle = c\cdot 0 = 0.
$$
And $\langle 0, w \rangle = 0$ trivially.

Hence $W^{\perp}$ is a subspace of $V$.
\end{solution}

\item[(d)]
\begin{answer}
False.
\end{answer}
\begin{solution}
The basis must be orthonormal.
\end{solution}

\item[(e)]
\begin{answer}
True.
\end{answer}
\begin{solution}
It follows from definition immediately.
\end{solution}

\item[(f)]
\begin{answer}
False.
\end{answer}
\begin{solution}
An orthogonal set must be nonzero.
\end{solution}

\item[(g)]
\begin{answer}
True.
\end{answer}
\begin{solution}
It follows from the Corollary 2 of Theorem 6.3.
\end{solution}

\end{enumerate}
\end{Exercise}