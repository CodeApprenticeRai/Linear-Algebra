% === Exercise 6.7.1 ===
\begin{Exercise}
	\begin{enumerate}[(a)]
		\item[(a)]
		\begin{answer}
			False.
		\end{answer}
		\begin{solution}
			Consider $A = \begin{pmatrix}
			2 & 0 \\
			0 & 2
			\end{pmatrix}$ where eigenvalues are $2$.
			However, the singular value of $A$ is $4$.
		\end{solution}
		
		\item[(b)]
		\begin{answer}
			False.
		\end{answer}
		\begin{solution}
			See part(a).
		\end{solution}
		
		\item[(c)]
		\begin{answer}
			True.
		\end{answer}
		\begin{solution}
			If $\sigma$ is the singular value of $A$, then $\sigma^2$ is the eigenvalue of $A^*A$.
			We know $(cA)^*(cA) = (c^*c)A^*A = c^2A^*A$ and hence $|c|^2 \sigma^2$ is the eigenvalue of $cA$.
			It follows that $|c|\sigma$ is the singular value of $cA$.
		\end{solution}
		
		\item[(d)]
		\begin{answer}
			True.
		\end{answer}
		\begin{solution}
			It follows from definition.
		\end{solution}
		
		\item[(e)]
		\begin{answer}
			False.
		\end{answer}
		\begin{solution}
			See part(a).
		\end{solution}
		
		\item[(f)]
		\begin{answer}
			False.
		\end{answer}
		\begin{solution}
			It follows from Theorem 6.30.
		\end{solution}
		
		\item[(g)]
		\begin{answer}
			True.
		\end{answer}
		\begin{solution}
			It follows from definition.		
		\end{solution}
	\end{enumerate}
\end{Exercise}