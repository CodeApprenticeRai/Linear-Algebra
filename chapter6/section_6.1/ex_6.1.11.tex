% === Exercise 6.1.11 ===
\begin{Exercise}
	\begin{proof}
		Consider for all $x,y\in V$,
		\begin{align*}
		\| x+y \|^2 + \| x-y \|^2
		&= \left(\sqrt{\langle x+y, x+y \rangle}\right)^2 + \left(\sqrt{\langle x-y, x-y \rangle}\right)^2 \\
		&= \langle x+y, x+y \rangle + \langle x-y, x-y \rangle \\
		&= (\langle x,x+y \rangle + \langle y,x+y \rangle) + (\langle x,x-y \rangle - \langle y,x-y \rangle) \\
		&= (\langle x,x+y \rangle + \langle x, x-y \rangle) + (\langle y,x+y \rangle + \langle y,x-y \rangle) \\
		&= \langle x,2x \rangle + \langle y,2y \rangle \\
		&= 2\langle x,x \rangle + 2\langle y,y \rangle \\
		&= 2\|x\|^2 + 2\|y\|^2.
		\end{align*}
		This means that the sum of square of the four edges is
		the sum of square of the two diagonals in the same parallelogram.
	\end{proof}
\end{Exercise}