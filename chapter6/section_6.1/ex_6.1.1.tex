% === Exercise 6.1.1 ===
\begin{Exercise}
\begin{enumerate}[(a)]
\item[(a)]
\begin{answer}
False.
\end{answer}
\begin{solution}
It follows from definition immediately.
\end{solution}

\item[(b)]
\begin{answer}
True.
\end{answer}
\begin{solution}
Since the field $F$ means $\mathbb{R}$ or $\mathbb{C}$.
\end{solution}

\item[(c)]
\begin{answer}
False.
\end{answer}
\begin{solution}
It is conjugate linear in the second component by Theorem 6.1.
\end{solution}

\item[(d)]
\begin{answer}
False.
\end{answer}
\begin{solution}
An inner product can be defined by our own. e.g., $\langle u,v\rangle := u v$.
\end{solution}

\item[(e)]
\begin{answer}
False.
\end{answer}
\begin{solution}
It follows from Theorem 6.2.
\end{solution}

\item[(f)]
\begin{answer}
False.
\end{answer}
\begin{solution}
It follows from definition immediately.
\end{solution}

\item[(g)]
\begin{answer}
False.
\end{answer}
\begin{solution}
Let $x = (1,1)$, $y = (1, 0)$, $z=(0,1)$ in $\mathbb{R}^2$. Then $\langle x, y \rangle = \langle x,z \rangle = 1$. However, $y\neq z$.
\end{solution}

\item[(h)]
\begin{answer}
True.
\end{answer}
\begin{solution}
Suppose to contrary that $y\neq 0$. By Theorem 6.1, $\langle x, y \rangle = 0$ implies $x = 0$ or $x = y$. Since $\langle x,y \rangle = 0$ for all $x$, this contradicts. Hence $y = 0$.
\end{solution}

\end{enumerate}
\end{Exercise}