% === Exercise 6.4.6 ===
\begin{Exercise}
\begin{enumerate}[(a)]
\item
\begin{proof}
Consider
$$
T_1^*
= \left[ \frac{1}{2}(T+T^*) \right]^*
= \frac{1}{2} (T+T^*)^*
= \frac{1}{2} (T^*+T)
= \frac{1}{2} (T + T^*)
= T_1.
$$
This means $T_1$ is self-adjoint.

Consider
$$
T_2^*
= \left[ \frac{1}{2}\ii(T-T^*) \right]^*
= \frac{1}{-2\ii}(T-T^*)^*
= -\frac{1}{2\ii}(T^*-T)
= \frac{1}{2\ii}(T-T^*)
= T_2.
$$
This means $T_2$ is self-adjoint.

Consider
$$
T_1 + \ii T_2
= \frac{1}{2}(T+T^*) + \ii \cdot \frac{1}{2\ii} (T-T^*)
= \frac{1}{2}T+ \frac{1}{2}T^* + \frac{1}{2}T - \frac{1}{2}T^*
= T.
$$
We obtain the desired result.
\end{proof}

\item
\begin{proof}
Suppose $T = U_1+\ii U_2$ with $U_1 = U_1^*$ and $U_2 = U_2^*$.

Consider
$$
T_1
= \frac{1}{2}(T+T^*)
= \frac{1}{2}(U_1 + \ii U_2 + U_1^* - \ii U_2^*)
= \frac{1}{2}(U_1 + \ii U_2^* + U_1 - \ii U_2^*)
= U_1.
$$
On the other hand,
$$
T_2
= \frac{1}{2\ii}(T-T^*)
= \frac{1}{2\ii}(U_1 + \ii U_2 - U_1^* + \ii U_2^*)
= \frac{1}{2\ii}(U_1^*+\ii U_2 - U_1^* + \ii U_2)
= U_2.
$$

As a result, we obtain the desired results.
\end{proof}

\item
\begin{proof}
$(\Longrightarrow)$
Since $T$ is normal, then $T T^* = T^*T$. Consider
\begin{align*}
T_1 T_2 
&= \frac{1}{2}(T+T^*)\cdot \frac{1}{2\ii}(T-T^*)
= \frac{1}{4\ii}(T^2+T^* T-T T^*-T)
= \frac{1}{4\ii}(T^2-T); \\
T_2 T_1
&= \frac{1}{2\ii}(T-T^*)\cdot \frac{1}{2}(T+T^*)
= \frac{1}{4\ii}(T^2-T^*T+T T^*-T)
= \frac{1}{4\ii}(T^2-T).
\end{align*}
It follows that $T_1 T_2 = T_2 T_1$.

\vspace{2ex}

$(\Longleftarrow)$
Compute $T_1 T_2$ and $T_2 T_1$, since $T_1 T_2 = T_2 T_1$, we have
$$
\frac{1}{4\ii}(T^2+T^* T-T T^*-T)
= \frac{1}{4\ii}(T^2-T^*T+T T^*-T).
$$
This implies
$$
2T^* T = 2T T^*.
$$
Then $T^* T = T T^*$ follows and hence $T$ is normal.
\end{proof}
\end{enumerate}
\end{Exercise}