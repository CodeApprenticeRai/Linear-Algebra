% === Exercise 6.4.1 ===
\begin{Exercise}
\begin{enumerate}[(a)]
\item[(a)]
\begin{answer}
True.
\end{answer}
\begin{solution}
Since $T = T^*$, then $T T^* = T^2 = T^* T$. This means $T$ is normal.
\end{solution}

\item[(b)]
\begin{answer}
False.
\end{answer}
\begin{solution}
Consider $T = \begin{pmatrix}
3 & 4 \\
1 & 0
\end{pmatrix}$, this implies $T$ have eigenvectors $\begin{pmatrix}
4 \\
1
\end{pmatrix}$, $\begin{pmatrix}
1 \\
-1
\end{pmatrix}$. On the other hand, $T^* = \begin{pmatrix}
3 & 1 \\
4 & 0
\end{pmatrix}$. This implies $T^*$ have eigenvectors $\begin{pmatrix}
1 \\
1
\end{pmatrix}$, $\begin{pmatrix}
-1 \\
4
\end{pmatrix}$. Hence their eigenvectors are different.
\end{solution}

\item[(c)]
\begin{answer}
False.
\end{answer}
\begin{solution}
$\beta$ should be an orthonormal basis. It follows from the definition of normal operator and Theorem 6.10.
\end{solution}

\item[(d)]
\begin{answer}
True.
\end{answer}
\begin{solution}
It follows from Theorem 6.10.
\end{solution}

\item[(e)]
\begin{answer}
True.
\end{answer}
\begin{solution}
It follows from the Lemma in page 373.
\end{solution}

\item[(f)]
\begin{answer}
True.
\end{answer}
\begin{solution}
Since $I^* = I$ and $O^* = O$.
\end{solution}

\item[(g)]
\begin{answer}
False.
\end{answer}
\begin{solution}
Consider $T = \begin{pmatrix}
0 & -1 \\
1 & 0
\end{pmatrix}$, then $T$ is a normal operator. However, $T$ is not diagonalizable.
\end{solution}

\item[(h)]
\begin{answer}
True.
\end{answer}
\begin{solution}
If the inner product space is over $\mathbb{R}$, then it follows from Theorem 6.17; if it is over $\mathbb{C}$, it follows from Theorem 6.16.
\end{solution}

\end{enumerate}
\end{Exercise}