% === Exercise 1.7.1 ===
\begin{Exercise}
	\begin{enumerate}[(a)]
		\item[(a)]
		\begin{answer}
			False.
		\end{answer}
		\begin{solution}
			For a counter-example, the family $\{1,2,\cdots,n,\cdots\}$ ordered by $\leq$ has no maximal element.
		\end{solution}
		
		\item[(b)]
		\begin{answer}
			False.
		\end{answer}
		\begin{solution}
			For a counter-example, the chain $\{1,2,\cdots,n,\cdots\}$ ordered by $\leq$ has no maximal element.
		\end{solution}
		
		\item[(c)]
		\begin{answer}
			False.
		\end{answer}
		\begin{solution}
			The family
			$$
			S = \{ \{a\}, \{b\} \}
			$$
			is ordered by containment. $S$ has two maximal elements $\{a\}$ and $\{b\}$.
		\end{solution}
		
		\item[(d)]
		\begin{answer}
			True.
		\end{answer}
		\begin{solution}
			Suppose to contrary that there exists at least two distinct maximal elements $A,B$. Since they are in a chain, so by definition, $A\subseteq B$ or $B\subseteq A$ must holds. Moreover, they are maximal, so $A\subseteq B$ and $B\subseteq A$. It follows that $A=B$ contradicts that $A,B$ are distinct.
		\end{solution}
		
		\item[(e)]
		\begin{answer}
			True.
		\end{answer}
		\begin{solution}
			Let $\beta$ be a basis for a vector space $V$. By definition of bases, $\beta$ is linearly independent. Also $\spann(\beta)=V$, by Theorem 1.7, if $v\in V$ and $v\notin \beta$, then $\beta\cup\{v\}$ is linearly dependent. That is, $\beta$ is a maximal linearly independent subset by definition.
		\end{solution}
		
		\item[(f)]
		\begin{answer}
			True.
		\end{answer}
		\begin{solution}
			It follows from Theorem 1.12.
		\end{solution}
		
	\end{enumerate}
\end{Exercise}