% === Exercise 1.1.7 ===
\begin{Exercise}
	\begin{proof}
		Consider $\parallelogram A B C D$. Notice that $\overrightarrow{B C} = \overrightarrow{A D}$ since the property of the parallelogram. Then
		\begin{align*}
			\overline{A C} 
			&= \left\{ t\overrightarrow{A C}: 0\leq t\leq 1 \right\} 
			= \left\{ t\left( \overrightarrow{A B} + \overrightarrow{B C} \right): 0\leq t\leq 1 \right\}
			= \left\{ t\left( \overrightarrow{A B} + \overrightarrow{A D} \right): 0\leq t\leq 1 \right\}; \\
			\overline{B D} 
			&= \left\{ \overrightarrow{A B} + s\left( \overrightarrow{A D} - \overrightarrow{AB} \right): 0\leq s\leq 1 \right\}.
		\end{align*}
		Hence the intersection $M$ of $\overline{A C}$ and $\overline{B D}$ is
		\begin{alignat*}{7}
			\quad&& t\left( \overrightarrow{A B} + \overrightarrow{A D} \right)
			&= \overrightarrow{A B} + s\left( \overrightarrow{A D} - \overrightarrow{AB} \right) \\
			\implies&& (1-s-t)\overrightarrow{A B} &= (t-s)\overrightarrow{A D}.
		\end{alignat*}
		Since $\overrightarrow{A B}\nparallel \overrightarrow{A D}$, then
		$$
		1-s-t = 0 \text{ and } t-s=0;
		$$
		therefore 
		$$
		s=t=\frac{1}{2}.
		$$
		Notice that $0\leq s,t \leq 1$, so $M$ exists.
		
		Denote $A = (x_a, y_a)$, $B = (x_b, y_b)$, $D = (x_d, y_d)$.
		We know the coordinates of $M$ is 
		$$
		A+\frac{1}{2}\overrightarrow{A C}
		= \left( \frac{x_c-x_a}{2}, \frac{y_c-y_a}{2} \right).
		$$
		By Exercise 1.1.6, we know $M$ bisects $\overline{A C}$. A similar argument establishes $M$ bisects $\overline{B D}$.
		
		Finally, we conclude the diagonals of a parallelogram bisect each other.
	\end{proof}
\end{Exercise}