% === Exercise 1.4.15 ===
\begin{Exercise}
\begin{proof}
Let $v\in span(S_1\cap S_2)$ arbitrarily, then we pick scalars $a_i$ and vectors $x_i\in S_1\cap S_2$ for each $i=1,2,\cdots,n$. It follows that
$$
v = \sum_{i=1}^{n} a_i x_i.
$$
So $v\in span(S_1)$ and $v\in span(S_2)$. Then $v\in span(S_1)\cap span(S_2)$.

By the arbitrary choice of $v$, we conclude $span(S_1\cap S_2) \subseteq span(S_1)\cap span(S_2)$.
\end{proof}
\begin{itemize}
\item $Example\ of\ \mathbf{span(S_1\cap S_2) = span(S_1)\cap span(S_2)}$.
\end{itemize}
\begin{answer}
$S_1 = S_2 = (1,1)$
\end{answer}
\begin{solution}
Since
$$
span(S_1\cap S_2) = \left\{ (x,y):x=y \right\},
$$
and
$$
span(S_1)\cap span(S_2) = \left\{ (x,y):x=y) \right\} \cap \left\{ (x,y):x=y) \right\} = \left\{ (x,y):x=y) \right\}.
$$
Hence 
$$
span(S_1\cap S_2) = span(S_1)\cap span(S_2).
$$
\end{solution}

\begin{itemize}
\item $Example\ of\ \mathbf{span(S_1\cap S_2) \neq span(S_1)\cap span(S_2)}$.
\end{itemize}
\begin{answer}
$S_1 = (0,1)$ and $S_2 = (1,0)$
\end{answer}
\begin{solution}
Since
$$
span(S_1\cap S_2) = span(\emptyset) = \left\{ (0,0) \right\},
$$
and
$$
span(S_1)\cap span(S_2) = \mathbb{R}^2.
$$
Hence 
$$
span(S_1\cap S_2) \neq span(S_1)\cap span(S_2).
$$
\end{solution}

\end{Exercise}