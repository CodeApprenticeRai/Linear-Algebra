% === Exercise 1.4.5 ===
\begin{Exercise}
	\begin{enumerate}[(a)]
		\item[(a)]
		\begin{answer}
			Yes.
		\end{answer}
		\begin{solution}
			Consider
			$$
			a(1,0,2)+b(-1,1,1) = (2,-1,1).
			$$
			This implies
			$$
			\begin{cases}
			\begin{aligned}
			a-b &= 2 \\
			b &= -1 \\
			2a+b &= 1
			\end{aligned}
			\end{cases}.
			$$
			Solve to obtain
			$$
			\begin{cases}
			\begin{aligned}
			a &= 1 \\
			b &= -1
			\end{aligned}
			\end{cases}.
			$$
			Hence $(2,-1,1) = (1,0,2)-(-1,1,1) \in span(S)$.
		\end{solution}
		
		\item[(b)]
		\begin{answer}
			No.
		\end{answer}
		\begin{solution}
			Consider
			$$
			a(1,0,2)+b(-1,1,1) = (-1,2,1).
			$$
			This implies
			$$
			\begin{cases}
			\begin{aligned}
			a-b &= -1 \\
			b &= 2 \\
			2a+b &= 1
			\end{aligned}
			\end{cases}.
			$$
			This systems of linear equations has no solution. Hence $(-1,2,1)\notin span(S)$.
		\end{solution}
		
		\item[(c)]
		\begin{answer}
			No.
		\end{answer}
		\begin{solution}
			Consider
			$$
			a(1,0,1,-1)+b(0,1,1,1) = (-1,1,1,2).
			$$
			This implies
			$$
			\begin{cases}
			\begin{aligned}
			a &= -1 \\
			b &= 1 \\
			a+b &= 1 \\
			-a+b &= 2
			\end{aligned}
			\end{cases}.
			$$
			This systems of linear equations has no solution. Hence $(-1,1,1,2)\notin span(S)$.
		\end{solution}
		
		\item[(d)]
		\begin{answer}
			Yes.
		\end{answer}
		\begin{solution}
			Consider
			$$
			a(1,0,1,-1)+b(0,1,1,1) = (2,-1,1,-3).
			$$
			This implies
			$$
			\begin{cases}
			\begin{aligned}
			a &= 2 \\
			b &= -1 \\
			a+b &= 1 \\
			-a+b &= -3
			\end{aligned}
			\end{cases}.
			$$
			Hence $(2,-1,1,-3) = 2(1,0,1,-1)-(0,1,1,1) \in span(S)$.
		\end{solution}
		
		\item[(e)]
		\begin{answer}
			Yes.
		\end{answer}
		\begin{solution}
			Consider
			$$
			a(x^3+x^2+x+1)+b(x^2+x+1)+c(x+1) = -x^3+2x^2+3x+3.
			$$
			This implies
			$$
			\begin{cases}
			\begin{aligned}
			a &= -1 \\
			a+b &= 2 \\
			a+b+c &= 3 \\
			a+b+c &= 3
			\end{aligned}
			\end{cases}.
			$$
			Solve to obtain $a=-1,b=3,c=1$. Hence 
			$$
			-x^3+2x^2+3x+3=-(x^3+x^2+x+1)+3(x^2+x+1)+(x+1)\in span(S).
			$$
		\end{solution}
		
		\item[(f)]
		\begin{answer}
			No.
		\end{answer}
		\begin{solution}
			Consider
			$$
			a(x^3+x^2+x+1)+b(x^2+x+1)+c(x+1) = 2x^3-x^2+x+3.
			$$
			This implies
			$$
			\begin{cases}
			\begin{aligned}
			a &= 2 \\
			a+b &= -1 \\
			a+b+c &= 1 \\
			a+b+c &= 3
			\end{aligned}
			\end{cases}.
			$$
			This system of linear equations have no solution. Hence $2x^3-x^2+x+3\notin \spann(S)$.
		\end{solution}
		
		\item[(g)]
		\begin{answer}
			Yes.
		\end{answer}
		\begin{solution}
			Consider
			$$
			a\begin{pmatrix}
			1 & 0 \\
			-1 & 0
			\end{pmatrix} + b\begin{pmatrix}
			0 & 1 \\
			0 & 1
			\end{pmatrix} + c\begin{pmatrix}
			1 & 1 \\
			0 & 0
			\end{pmatrix} = \begin{pmatrix}
			1 & 2 \\
			-3 & 4
			\end{pmatrix}.
			$$
			This implies
			$$
			\begin{cases}
			\begin{aligned}
			a+c &= 1 \\
			b+c &= 2 \\
			-a &= -3 \\
			b &= 4
			\end{aligned}
			\end{cases}.
			$$
			Solve to obtain $a=3,b=4,c=-2$. Hence
			$
			\begin{pmatrix}
			1 & 2 \\
			-3 & 4
			\end{pmatrix} \in \spann(S).
			$
		\end{solution}
		
		\item[(h)]
		\begin{answer}
			No.
		\end{answer}
		\begin{solution}
			Consider
			$$
			a\begin{pmatrix}
			1 & 0 \\
			-1 & 0
			\end{pmatrix} + b\begin{pmatrix}
			0 & 1 \\
			0 & 1
			\end{pmatrix} + c\begin{pmatrix}
			1 & 1 \\
			0 & 0
			\end{pmatrix} = \begin{pmatrix}
			1 & 0 \\
			0 & 1
			\end{pmatrix}.
			$$
			This implies
			$$
			\begin{cases}
			\begin{aligned}
			a+c &= 1 \\
			b+c &= 0 \\
			-a &= 0 \\
			b &= 1
			\end{aligned}
			\end{cases}.
			$$
			This system of linear equations have no solution. Hence 
			$
			\begin{pmatrix}
			1 & 0 \\
			0 & 1
			\end{pmatrix} \notin \spann(S).
			$
		\end{solution}
	\end{enumerate}
\end{Exercise}