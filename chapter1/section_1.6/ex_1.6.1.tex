% === Exercise 1.6.1 ===
\begin{Exercise}
	\begin{enumerate}[(a)]
		\item[(a)]
		\begin{answer}
			False.
		\end{answer}
		\begin{solution}
			Since $span(\emptyset)=\{0\}$, then $\emptyset$ is a basis for the zero vector space.
		\end{solution}
		
		\item[(b)]
		\begin{answer}
			True.
		\end{answer}
		\begin{solution}
			It follows from Theorem 1.9. Moreover, bases must be finite.
		\end{solution}
		
		\item[(c)]
		\begin{answer}
			False.
		\end{answer}
		\begin{solution}
			The counter-example is $P(F)$, which has a infinite basis.
		\end{solution}
		
		\item[(d)]
		\begin{answer}
			False.
		\end{answer}
		\begin{solution}
			Here we give a counter-example. $\mathbb{R}^2$ has a basis $\{(0,1), (1,0)\}$; also, it has a basis $\{(1,1),(0,1)\}$.
		\end{solution}
		
		\item[(e)]
		\begin{answer}
			True.
		\end{answer}
		\begin{solution}
			It follows from the Corollary of Theorem 1.10.
		\end{solution}
		
		\item[(f)]
		\begin{answer}
			False.
		\end{answer}
		\begin{solution}
			$\dim\left(P_n(F)\right) = \mathbf{n+1}$.
		\end{solution}
		
		\item[(g)]
		\begin{answer}
			False.
		\end{answer}
		\begin{solution}
			$\dim\left(M_{n\times n(F)}\right) = \mathbf{m n}$.
		\end{solution}
		
		\item[(h)]
		\begin{answer}
			True.
		\end{answer}
		\begin{solution}
			It follows from the Replacement Theorem directly.
		\end{solution}
		
		\item[(i)]
		\begin{answer}
			False.
		\end{answer}
		\begin{solution}
			Here we give a counter-example. Let $S := \left\{(0,1),(0,2),(1,1)\right\}$, then $span(S) = \mathbb{R}^2$. However, $(0,5) = 3(0,1)+(0,2) = (0,1)+2(0,2)\in\mathbb{R}^2$.
		\end{solution}
		
		\item[(j)]
		\begin{answer}
			True.
		\end{answer}
		\begin{solution}
			Let $W$ is a subspace of $V$ where $\dim(V) < \infty$, then by Theorem 1.11, we know $\dim(W)\leq \dim(V)$; hence $\dim(W) < \infty$.
		\end{solution}
		
		\item[(k)]
		\begin{answer}
			True.
		\end{answer}
		\begin{solution}
			We have known $\dim(V) = n$ by hypothesis. Then $\{0\}$ is the only subspace of $V$ such that $\dim(\{0\}) = 0$, and $V$ is the only subspace of $V$ such that $\dim(V) = n$.
		\end{solution}
		
		\item[(l)]
		\begin{answer}
			True.
		\end{answer}
		\begin{solution}
			It follows from the Corollary 2 of Theorem 1.10.
		\end{solution}
		
	\end{enumerate}
\end{Exercise}