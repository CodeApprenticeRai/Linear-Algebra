% === Exercise 1.6.29 ===
\begin{Exercise}
	\begin{enumerate}[(a)]
		\item
		\begin{proof}
			Follow the hint. First, we claim $\beta = \{u_1,\cdots,u_k,v_1,\cdots,v_m,w_1,\cdots,w_p\}$ is a basis for $W_1+W_2$. 
			\begin{itemize}
				\item $\mathbf{\boldsymbol{\beta}\ generates\ W_1+W_2}$.
			\end{itemize}
			Let 
			\begin{align*}
			t_1 &:= \sum_{i=1}^{k}a_i u_i + \sum_{i=1}^{m}b_i v_i\in W_1; \\
			t_2 &:= \sum_{i=1}^{k}a'_i u_i + \sum_{i=1}^{p}c_i w_i\in W_2.
			\end{align*}
			Then pick $t = t_1+t_2 \in W_1+W_2$, we have 
			$$
			t = t_1+t_2 = \sum_{i=1}^{k}(a_i+a'_i) u_i + \sum_{i=1}^{m}b_i v_i + \sum_{i=1}^{p}c_i w_i \in span(\beta).
			$$
			Since $t_1$ and $t_2$ are arbitrary, this implies $W_1+W_2 =  span(\beta)$.
			
			\begin{itemize}
				\item $\mathbf{\boldsymbol{\beta}\ is\ linearly\ independent}$.
			\end{itemize}
			Consider
			\begin{equation}\label{eq:ex_1.6.29a1}
			\sum_{i=1}^{k}a_i u_i + \sum_{i=1}^{m}b_i v_i + \sum_{i=1}^{p}c_i w_i = 0.
			\end{equation}
			Let
			\begin{equation}\label{eq:ex_1.6.29a2}
			v := \sum_{i=1}^{k}a_i u_i \sum_{i=1}^{m}b_i v_i = -\sum_{i=1}^{p}c_i w_i.
			\end{equation}
			We know $v\in W_1$ and $v\in W_2$; hence $v\in W_1\cap W_2$. It follows that
			$$
			v = \sum_{i=1}^{k}a'_i u_i.
			$$
			Since $\{u_1, \cdots, u_k, w_1, \cdots, w_p\}$ is a basis for $W_2$ and $v\in W_2$, then
			$$
			\sum_{i=1}^{k}a'_i u_i + \sum_{i=1}^{p}c_i w_i = 0
			\implies
			a'_1 = \cdots = a'_k = c_1 = \cdots = c_p = 0.
			$$
			Also since $\{u_1, \cdots, u_k, v_1, \cdots, v_m\}$ is a basis for $W_1$ and $v\in W_1$, from equation \eqref{eq:ex_1.6.29a2}, we know
			$$
			v = \sum_{i=1}^{k}a_i u_i \sum_{i=1}^{m}b_i v_i = 0
			\implies
			a_1 = \cdots = a_k = b_1 = \cdots = b_m = 0.
			$$
			Hence from equation \eqref{eq:ex_1.6.29a1}, we obtain $\beta$ is linearly independent.
			
			We prove the claim holds as promised. Finally, we compute
			\begin{align}
			\dim(W_1+W_2) 
			&= k+m+p \notag \\
			&= (k+m) + (k+p) - k \notag \\
			&= \dim(W_1) + \dim(W_2) - \dim(W_1\cap W_2). \label{eq:ex_1.6.29a3}
			\end{align}
			
			By hypothesis that $W_1,W_2$ are finite-dimensional vector spaces, we know $W_1\cap W_2$ is also finite-dimensional; hence we obtain $k+m,k+p,k<\infty$. From formula \eqref{eq:ex_1.6.29a3}, we conclude $W_1+W_2$ is also a finite-dimensional vector space since $k+m+p<\infty$.
		\end{proof}
		
		\item
		\begin{proof}
			From part (a) and by hypothesis, we know
			$$
			\dim(W_1+W_2) = \dim(W_1)+\dim(W_2)-\dim(W_1\cap W_2).
			$$
			Hence
			\begin{alignat*}{7}
			\quad&& V &= W_1\oplus W_2 \\
			\iff&& V &= W_1+W_2\text{ and }W_1\cap W_2=\{0\} \\
			\iff&& \dim(V) &= \dim(W_1)+\dim(W_2)-\dim(\{0\}) \\
			\iff&& \dim(V) &= \dim(W_1)+\dim(W_2)
			\end{alignat*}
		\end{proof}
	\end{enumerate}
\end{Exercise}