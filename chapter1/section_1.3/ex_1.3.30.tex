% === Exercise 1.3.30 ===
\begin{Exercise}
\begin{proof}
$(\Longrightarrow)$
We have known $V = W_1\oplus W_2$. Let $x$ be an arbitrary vector in $V$. Suppose to contrary that $x$ cannot be uniquely written. i.e.,
$$
x = x_1+x_2 = x'_1+x'_2
$$
where $x_1, x'_1\in W_1$ and $x_2, x'_2\in W_2$.

Notice that $x_1-x'_1 = -(x_2-x'_2)$. Since $x_1-x'_1\in W_1$ and $-(x_2-x'_2)\in W_2$ from the property of vector spaces in $W_1$ and $W_2$, we know $x_1-x'_1\in W_2$ and $-(x_2-x'_2)\in W_1$. Because $W_1\cap W_2 = \{0\}$, $x_1-x'_1=0$ and $-(x_2-x'_2)=0$. It follows that
$$
x_1 = x'_1\text{ and } x_2 = x'_2.
$$
This contradicts the supposition. Hence $x$ can be uniquely written as what we need to show.

By the arbitrary choice of $x$, we conclude for each vector in $V$ can be uniquely written as $x_1+x_2$ where $x_1\in W_1$ and $x_2\in W_2$.

\vspace{2ex}

$(\Longleftarrow)$
We have known for each vector in $V$ can be uniquely written as $x_1+x_2$ where $x_1\in W_1$ and $x_2\in W_2$. This follows $V = W_1+W_2$ directly.

Suppose to contrary that $W_1\cap W_2\neq \{0\}$. Let $x\in V$ such that $x\neq 0$ and $x\in W_1\cap W_2$. Then
$$
x = x + 0_{W_1} = 0_{W_2} + x
$$
where $0_{W_1}\in W_1$ and $0_{W_2}\in W_2$. This contradict the hypothesis; hence $W_1\cap W_2= \{0\}$.

We together the results $W_1+W_2=V$ and $W_1\cap W_2=\{0\}$ to obtain $V=W_1\oplus W_2$.
\end{proof}
\end{Exercise}