% === Exercise 1.3.1 ===
\begin{Exercise}
\begin{enumerate}[(a)]
\item[(a)]
\begin{answer}
False.
\end{answer}
\begin{solution}
For example, let $V := \mathbb{R}$ and $W := \mathbb{Q}$, then $W$ is a subset of $V$ and $V$ is a vector space over $\mathbb{R}$; however, $W$ is not a vector space over $\mathbb{R}$.
\end{solution}

\item[(b)]
\begin{answer}
False.
\end{answer}
\begin{solution}
$0$ is not an element of $\emptyset$ and hence $\emptyset$ is not a subspace of every vector space.
\end{solution}

\item[(c)]
\begin{answer}
True.
\end{answer}
\begin{solution}
We pick $W$ such as a zero vector space.
\end{solution}

\item[(d)]
\begin{answer}
False.
\end{answer}
\begin{solution}
Let $V = \mathbb{R}$, then we pick $W_0 = \{0\}$ and $W_1 = \{1\}$. We know $W_0$, $W_1$ are subsets of $V$, however $W_0\cap W_1 = \emptyset$ is not a subspace of $V$ form part (b).
\end{solution}

\item[(e)]
\begin{answer}
True.
\end{answer}
\begin{solution}
We know only diagonal entries could be $0$ by definition of a diagonal matrix. Since this is $n\times n$ diagonal matrix, it contains at most $n$ zero entries.
\end{solution}

\item[(f)]
\begin{answer}
False.
\end{answer}
\begin{solution}
The trace of a square matrix is the $\mathbf{sum}$ of its diagonal entries.
\end{solution}

\item[(g)]
\begin{answer}
False.
\end{answer}
\begin{solution}
Since $(0,0,0)\in W$, however $(0,0,0)\notin \mathbb{R}^2$.
\end{solution}

\end{enumerate}
\end{Exercise}