% === Exercise 1.3.23 ===
\begin{Exercise}
\begin{enumerate}[(a)]
\item
\begin{proof}
Let $a := x_1+y_1, b := x_2+y_2\in W_1+W_2$ where $x_1,x_2\in W_1$ and $y_1,y_2\in W_2$. Also let $c$ be a arbitrary scalar. Let $0_{W_1}$, $0_{W_2}$ be zero vectors in $W_1$, $W_2$, respectively. Then
$$
a+b
= (x_1+y_1) + (x_2+y_2)
= (x_2+y_2) + (x_1+y_1)
= b+a \in W_1+W_2.
$$
Also,
$$
c (a+b)
= c (x_1+y_1+x_2+y_2)
= c (x_1+y_1) + c(x_2+y_2)
= c a + c b \in W_1+W_2.
$$
Moreover,
$$
0_{W_1} + 0_{W_2} \in W_1+W_2.
$$
Hence we obtain $W_1+W_2$ is a subspace of $V$.

Furthermore, by definition, we know
\begin{align*}
W_1+W_2
&= \{x+y:x\in W_1\text{ and }y\in W_2\} \\
&\supseteq \{x+0_{W_2}:x\in W_1\text{ and } 0_{W_2}\in W_2\} \\
&= W_1.
\end{align*}
Also,
\begin{align*}
W_1+W_2
&= \{x+y:x\in W_1\text{ and }y\in W_2\} \\
&\supseteq \{0_{W_1}+y:0_{W_1}\in W_1\text{ and } y\in W_2\} \\
&= W_2.
\end{align*}
Hence $W_1+W_2$ contains both $W_1$ and $W_2$.
\end{proof}

\item 
\begin{proof}
Let $W$ be an arbitrary subspace of $V$ such that $W\supseteq W_1$ and $W\supseteq W_2$. Therefore, for all $x\in W_1$ and $y\in W_2$ imply $x,y\in W$.

Consider
\begin{alignat*}{7}
\quad&& W_1+W_2 &= \left\{x+y:x\in W_1\text{ and }y\in W_2\right\} \\
\implies&& W_1+W_2 &\subseteq \left\{x+y:x\in W\text{ and }y\in W\right\} \\
\implies&& W_1+W_2 &\subseteq W.
\end{alignat*}
By the arbitrary choice of $W$, we know any subspace of $V$ that contains $W_1$ and $W_2$ must also contain $W_1+W_2$ as promised.
\end{proof}
\end{enumerate}
\end{Exercise}