% === Exercise 1.3.8 ===
\begin{Exercise} \label{ex:1.3.8}
\begin{enumerate}[(a)]
\item[(a)]
\begin{answer}
Yes.
\end{answer}
\begin{solution}
Let $x := (3t,t,-t)\in W_1$ for $t\in\mathbb{R}$, and $y := (3k,k,-k)\in W_1$ for $k\in\mathbb{R}$, and $c$ is a  arbitrary scalar. 
Then
$$
x+y
= (3t+3k,t+k,-(t+k))
= (3(t+k), t+k, -(t+k)) \in W_1.
$$
Also,
$$
c x
= c(3t,t,-t)
= (3(ct),ct,-ct) \in W_1.
$$
Moreover,
$$
(0,0,0) \in W_1
$$
Hence $W_1$ is a subspace of $\mathbb{R}^3$.
\end{solution}

\item[(b)]
\begin{answer}
No.
\end{answer}
\begin{solution}
Let $x := (x_3+2,x_2,x_3)\in W_2$ and $y := (y_3+2,y_2,y_3)\in W_2$. 
Then
$$
x+y
= (x_3+y_3+4, x_2+y_2,x_3+y_2) \notin W_2.
$$
Hence $W_2$ is not a subspace of $\mathbb{R}^3$.
\end{solution}

\item[(c)]
\begin{answer}
Yes.
\end{answer}
\begin{solution}
Let $x := (x_1,x_2,-2x_1+7x_2)\in W_3$ and $y := (y_1,y_2,-2y_1+7y_2)\in W_3$ and $c$ is a arbitrary scalar. Then
$$
x+y
= (x_1+y_1, x_2+y_2, -2(x_1+y_1)+7(x_2+y_2)) \in W_3.
$$
And
$$
c x
= (c x_1, c x_2, c(-2 x_1 + 7x_2)) \in W_3.
$$
Moreover,
$$
(0,0,0) \in W_3.
$$
Hence $W_3$ is a subspace of $\mathbb{R}^3$.
\end{solution}

\item[(d)]
\begin{answer}
Yes.
\end{answer}
\begin{solution}
Let $x := (x_1,x_2,-x_1+4x_2)\in W_4$ and $y := (y_1,y_2,-y_1+4y_2)\in W_4$ and $c$ is a arbitrary scalar. Then
$$
x+y
= (x_1+y_1, x_2+y_2, -(x_1+y_1)+4(x_2+y_2)) \in W_4.
$$
And
$$
c x
= (c x_1, c x_2, c(- x_1 + 4x_2)) \in W_4.
$$
Moreover,
$$
(0,0,0) \in W_4.
$$
Hence $W_4$ is a subspace of $\mathbb{R}^3$.
\end{solution}

\item[(e)]
\begin{answer}
No.
\end{answer}
\begin{solution}
Since $(0,0,0)\notin W_5$, then $W_5$ is not a subspace of $\mathbb{R}^3$.
\end{solution}

\item[(f)]
\begin{answer}
No.
\end{answer}
\begin{solution}
Let $x := (1,\frac{\sqrt{33}}{3},1)\in W_6$ and $y := (1,1,\frac{\sqrt{3}}{3})\in W_6$, however $x+y\notin W_6$.
Since there exists two elements of $W_6$ such that the sum of them is not an element of $W_6$; hence $W_6$ is not a subspace of $\mathbb{R}^3$.
\end{solution}

\end{enumerate}
\end{Exercise}