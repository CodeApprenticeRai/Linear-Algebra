% === Exercise 1.2.1 ===
\begin{Exercise}
\begin{enumerate}[(a)]
\item [(a)]
\begin{answer}
True.
\end{answer}
\begin{solution}
This follows from the definition $(VS\ 3)$.
\end{solution}

\item [(b)]
\begin{answer}
False.
\end{answer}
\begin{solution}
This is by corollary 1 of Theorem 1.1. The proof could reference Exercise \ref{ex:1.2.9}.
\end{solution}

\item [(c)]
\begin{answer}
False.
\end{answer}
\begin{solution}
If $x = 0$, then $a$ might not equal to $b$.
\end{solution}

\item [(d)]
\begin{answer}
False.
\end{answer}
\begin{solution}
If $a = 0$, then $x$ might not equal to $y$.
\end{solution}

\item [(e)]
\begin{answer}
True.
\end{answer}
\begin{solution}
This is the definition.
\end{solution}

\item [(f)]
\begin{answer}
False.
\end{answer}
\begin{solution}
It has $m$ rows and $n$ columns.
\end{solution}

\item [(g)]
\begin{answer}
False.
\end{answer}
\begin{solution}
Since $x$ and $x^2$ are members of $P(F)$, then the sum of them is $x^2+x$; however, $x$ has degree $1$ and $x^2$ has degree $2$ which implies $x$ and $x^2$ have different degrees.
\end{solution}

\item [(h)]
\begin{answer}
False.
\end{answer}
\begin{solution}
$x$ and $-x$ are polynomials of degree $1$, but $x+(-x) = 0$ is a polynomial of degree $-1$.
\end{solution}

\item [(i)]
\begin{answer}
True.
\end{answer}
\begin{proof}
Let $f(x) = a_n x^n + \cdots + a_0$ is a polynomial of degree $n$, and $c$ is a nonzero scalar. Notice that $a_n \neq 0$. Consider 
$$
cf(x) = c a_n x^n + \cdots + c a_0.
$$
Since $c a_n \neq 0$, then $cf(x)$ is also a polynomial of degree $n$ by definition.
\end{proof}

\item [(j)]
\begin{answer}
True.
\end{answer}
\begin{solution}
This is by definition directly.
\end{solution}

\item [(k)]
\begin{answer}
True.
\end{answer}
\begin{solution}
This is the definition.
\end{solution}
\end{enumerate}
\end{Exercise}