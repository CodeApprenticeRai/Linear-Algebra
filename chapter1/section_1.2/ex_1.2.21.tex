% === Exercise 1.2.21 ===
\begin{Exercise}
\begin{proof}
It suffices to prove $Z$ is compatible with the definition from $(VS\ 1)$ to $(VS\ 8)$. We denoted the zero vector in  $V$ is $0_V$ and $W$ is $0_W$. Notice that we have known $V$, $W$ are compatible with the definitions since they are vector spaces by hypothesis.

Suppose $x := (x_1,x_2)$ and $y := (y_1,y_2)$ and $z := (z_1,z_2)$ and the zero vector in $Z$ is $0_Z := (0_V,0_W)$. Here we go by verifying patiently.
\begin{itemize}
\item $\mathbf{(VS\ 1)}$
For all $x,y \in Z$, we have
$$
x+y
= (x_1+y_1,x_2+y_2)
= (y_1+x_1,y_2+x_2)
= y+x.
$$

\item $\mathbf{(VS\ 2)}$
For all $x,y,z\in Z$, we have
$$
(x+y)+z
= ( (x_1+y_1)+z_1, (x_2+y_2)+z_2 )
= ( x_1+(y_1+z_1), x_2+(y_2+z_2) )
= x+(y+z).
$$

\item $\mathbf{(VS\ 3)}$
For all $x\in Z$, we have
$$
x+0_Z
= (x_1 + 0_V, x_2+0_W)
= (x_1,x_2)
= x.
$$

\item $\mathbf{(VS\ 4)}$
For all $x\in Z$, we pick $y' = (-x_1, -x_2)$ since $-x_1\in V$ and $-x_2\in W$. So $y'\in Z$. Then we have
$$
x+y'
= (x_1-x_1, x_2-x_2)
= 0.
$$

\item $\mathbf{(VS\ 5)}$
For all $x\in Z$, we have
$$
1x
= (1 x_1,1 x_2)
= (x_1, x_2)
= x.
$$

\item $\mathbf{(VS\ 6)}$
For all $a,b\in F$ and $x\in Z$, we have
\begin{align*}
(ab)x
&= ( (ab) x_1, (ab) x_2) \\
&= (a (b x_1), a (b x_2)) \\
&= a (b x_1, b x_2) \\
&= a (b (x_1, x_2) ) \\
&= a(bx).
\end{align*}

\item $\mathbf{(VS\ 7)}$
For all $a\in F$ and $x,y\in Z$, we have
\begin{align*}
a(x+y)
&= (a(x_1+y_1), a(x_2+y_2)) \\
&= (a x_1 + a y_1, a x_2 + a y_2) \\
&= (a x_1, a x_2) + (a y_1, a y_2) \\
&= a(x_1, x_2) + a(y_1,y_2) \\
&= a x+a y.
\end{align*}

\item $\mathbf{(VS\ 8)}$
For all $a,b\in F$ and $x\in Z$, we have
\begin{align*}
(a+b)x
&= ( (a+b) x_1, (a+b) x_2 ) \\
&= ( a x_1 + b x_1, a x_2 + b x_2 ) \\
&= (a x_1, a x_2) + (b x_1 + b x_2) \\
&= a(x_1, x_2) + b(x_1, x_2) \\
&= a x + b x.
\end{align*}
\end{itemize}
We conclude $Z$ is a vector space over $F$ as promised.
\end{proof}
\end{Exercise}