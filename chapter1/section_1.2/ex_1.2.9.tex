% === Exercise 1.2.9 ===
\begin{Exercise}\label{ex:1.2.9}
\begin{itemize}
\item $\mathbf{Prove\ Corollary\ 1\ of\ Theorem\ 1.1}$ 
\end{itemize}
\begin{proof}
We have known the zero vector always exists by definition $(VS\ 3)$. Suppose to contrary that a vector space contains more than one zero vector, so at least two zero vectors in this vector space. 

Suppose $x,y$ are arbitrary distinct zero vectors, then by $(VS\ 3)$, we know $x = x+y =y$ which means $x,y$ are identical. This contradicts $x,y$ are distinct.

Hence we conclude the zero vector in a vector space is unique.
\end{proof}

\begin{itemize}
\item $\mathbf{Prove\ Corollary\ 2\ of\ Theorem\ 1.1}$ 
\end{itemize}
\begin{proof}
We have known for every $x\in V$, there always exists an element $y\in V$ such that $x+y=0$ by definition $(VS\ 4)$. 

Suppose to contrary that at least two distinct elements $y,z$ such that $x+y=0$ and $x+z=0$, then we have $y=z$ which contradicts $y\neq z$ by the supposition.

Hence the element $y$ mentioned from definition $(VS\ 4)$ is unique.
\end{proof}

\begin{itemize}
\item $\mathbf{Prove\ Theorem\ 1.2(c)}$ 
\end{itemize}
\begin{proof}
By definition $(VS\ 1)$, $(VS\ 3)$, $(VS\ 7)$, we know for all $a\in F$,
$$
a0 + a0 = a(0+0) = a0 = a0+0 = 0+a0.
$$
Hence $a0 = 0$ for all $a\in F$.
\end{proof}

\end{Exercise}