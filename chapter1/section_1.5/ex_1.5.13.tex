% === Exercise 1.5.13 ===
\begin{Exercise}
	\begin{proof}
		$(\Longrightarrow)$
		Since $\{u,v\}$ is linearly independent, we consider
		$$
		a u+b v = 0.
		$$
		where $a=b=0$. Then we let
		$$
		a' := \frac{a+b}{2}\text{ and } b' := \frac{a-b}{2}.
		$$
		Notice that $a'=b'=0$. Consider
		\begin{align*}
		a' (u+v)+b' (u-v)
		&= \frac{a+b}{2}(u+v) + \frac{a-b}{2}(u-v) \\
		&= a u+b v \\
		&= 0.
		\end{align*}
		This means $\{u+v,u-v\}$ is linearly independent.
		
		\vspace{2ex}
		
		$(\Longleftarrow)$
		Since $\{u+v,u-v\}$ is linearly independent, we consider
		$$
		a (u+v)+b (u-v) = 0.
		$$
		where $a=b=0$. Then we let
		$$
		a' := a+b\text{ and } b' := a-b.
		$$
		Notice that $a'=b'=0$. Consider
		\begin{align*}
		a' u + b' v
		&= (a+b)u + (a-b)v \\
		&= a(u+v)+b(u-v) \\
		&= 0.
		\end{align*}
		This means $\{u,v\}$ is linearly independent.
	\end{proof}
\end{Exercise}