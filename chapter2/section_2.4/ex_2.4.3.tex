% === Exercise 2.4.3 ===
\begin{Exercise}
They follow from Theorem 2.19 immediately.
\begin{enumerate}[(a)]
\item[(a)]
\begin{answer}
No.
\end{answer}
\begin{solution}
Since $\dim(F^3) = 3 \neq 4 = \dim(P_3(F))$.
\end{solution}

\item[(b)]
\begin{answer}
Yes.
\end{answer}
\begin{solution}
Since $\dim(F^4) = 4 = \dim(P_3(F))$.
\end{solution}

\item[(c)]
\begin{answer}
Yes.
\end{answer}
\begin{solution}
Since $\dim(M_{2\times 2}(\mathbb{R})) = 4 = \dim(P_3(\mathbb{R}))$.
\end{solution}

\item[(d)]
\begin{answer}
No.
\end{answer}
\begin{solution}
Notice that $\{E_{1 1}+E_{2 2}, E_{1 2}, E_{2 1}\}$ is a basis for $V$. So $\dim(V) = 3 \neq 4 = \dim(\mathbb{R}^4)$.
\end{solution}

\end{enumerate}
\end{Exercise}