% === Exercise 2.1.1 ===
\begin{Exercise}
	\begin{enumerate}[(a)]
		\item[(a)]
		\begin{answer}
			True.
		\end{answer}
		\begin{solution}
			This is by definition.
		\end{solution}
		
		\item[(b)]
		\begin{answer}
			False.
		\end{answer}
		\begin{solution}
			Let $T:\mathbb{C}\to\mathbb{C}$ such that $T(a+b i) = a$, and $x := a_1 + b_1 i$ and $y := a_2 + b_2 i$.
			
			Notice that $T(x+y) = T(a_1 + b_1 i + a_2 + b_2 i) = a_1 + a_2 = T(x) + T(y)$. However, $T(c i) = 0$ and $i T(c) = c i$ are distinct. So $T$ is not linear.
		\end{solution}
		
		\item[(c)]
		\begin{answer}
			False.
		\end{answer}
		\begin{solution}
			We will give a counter-example for each direction.
			
			$(\Longrightarrow)$
			Let $T:\mathbb{R}\to\mathbb{R}$ such that $T(x) = x+1$. We know $T$ is one-to-one, however $T(x) = 0$ implies $x=-1$.
			
			\vspace{2ex}
			
			$(\Longleftarrow)$
			Let $T(x) = |x|$. Since $T(x) = 1$ implies $x = 1$ or $x=-1$, then $T$ is not one-to-one.
		\end{solution}
		
		\item[(d)]
		\begin{answer}
			True.
		\end{answer}
		\begin{solution}
			Since
			$$
			T(0_V) = T(0_V+0_V) = T(0_V) + T(0_V),
			$$
			then we conclude $T(0_V) = 0_W$.
		\end{solution}
		
		\item[(e)]
		\begin{answer}
			False.
		\end{answer}
		\begin{solution}
			It follows from Theorem 2.3. Notice that it equals to $\dim(V)$, not $\dim(W)$.
		\end{solution}
		
		\item[(f)]
		\begin{answer}
			False.
		\end{answer}
		\begin{solution}
			Consider $T:\mathbb{R}\to\mathbb{R}$ such that $T(x) = 0$.
		\end{solution}
		
		\item[(g)]
		\begin{answer}
			True.
		\end{answer}
		\begin{solution}
			It follows from the Corollary of Theorem 2.6.
		\end{solution}
		
		\item[(h)]
		\begin{answer}
			False.
		\end{answer}
		\begin{solution}
			It is precise to say this statement might hold or not hold. In other words, it doesn't hold always.
			
			For example, we set $x_2 = 2x_1$, then $T(x_2) = T(2 x_1) = 2T(x_1) = 2 y_1$. Hence if $y_2\neq 2 y_1$, there doesn't exist such a linear transformation.
		\end{solution}
	\end{enumerate}
\end{Exercise}