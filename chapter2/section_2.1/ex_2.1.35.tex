% === Exercise 2.1.35 ===
\begin{Exercise}
\begin{enumerate}[(a)]
\item
\begin{proof}
Notice that $V$ is finite-dimensional vector space. then both $nullity(T)$ and $rank(T)$ are finite. By hypothesis $V=R(T)+N(T)$, it suffices to prove $R(T)\cap N(T) = \{0\}$.

By Exercise 1.6.29(a), we know 
$$
dim(R(T)+N(T)) = dim(R(T))+dim(N(T))-dim(R(T)\cap N(T)).
$$
By hypothesis, we have
$$
dim(R(T)\cap N(T)) = dim(R(T))+dim(N(T))-dim(V).
$$
Since $dim(V)$ is finite, by the Dimension Theorem, we have
$$
dim(V) = rank(T)+ nullity(T) = dim(R(T))+dim(N(T)).
$$
Finally we obtain
$$
dim(R(T)\cap N(T)) = 0
$$
which implies
$$
R(T)\cap N(T) = \{0\}.
$$
Hence
$$
V = R(T)\oplus N(T).
$$
\end{proof}

\item
\begin{proof}
This is similar to part (a). Notice that $dim(V)$ is finite, then both $nullity(T)$ and $rank(T)$ are finite. Consider
$$
dim(R(T)+N(T)) = dim(R(T))+dim(N(T))-dim(R(T)\cap N(T)).
$$
By hypothesis, we know
$$
dim(R(T)+N(T)) = dim(R(T))+dim(N(T))-0.
$$
By the Dimension Theorem, we have
$$
dim(R(T)+N(T)) = dim(R(T))+dim(N(T)) = dim(V).
$$
Hence $R(T)+N(T) = V$. Then following $R(T)\cap N(T) = \{0\}$ by hypothesis, we conclude
$$
V = R(T)\oplus N(T).
$$
\end{proof}
\end{enumerate}
\end{Exercise}