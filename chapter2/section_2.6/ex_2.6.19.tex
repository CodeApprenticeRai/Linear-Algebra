% === Exercise 2.6.19 ===
\begin{Exercise}
	\begin{proof}
		Since $W\subset V$ but $W\neq V$ by hypothesis, we can pick $x_0\in V\setminus W$. Let $\beta$ be a basis for $W$. Since $x_0\notin W$, then we have $\beta\cup \{x_0\}$ is linearly independent. Now we extend $\beta\cup \{x_0\}$ to $\beta'$ which is a basis for $V$.
		
		Define a function $g:\beta'\to F$ by 
		$$
		g(x)=\begin{cases}
		1 & \mbox{ for} x=x_0 \\
		0 & \mbox{ otherwise}
		\end{cases}.
		$$
		Hence we know $g(x)=0$ for all $x\in \beta$.
		
		By Exercise 2.1.34, we know there exists the unique linear transformation $f:\beta'\to F$ such that $f(x)=g(x)$ for all $x\in \beta'$. Then, we can write $y\in W$ as $y=\sum_{j=1}^{n} c_j y_j$ where $y_j\in \beta$ and $c_j\in F$. Consider
		$$
		f(y) = f(\sum_{j=1}^{n} c_j y_j) 
		= \sum_{j=1}^{n} c_j f(x_j)
		= \sum_{j=1}^{n} c_j g(x_j)
		= 0
		$$
		for all $y\in W$.
		Notice that $f(x_0) = g(x_0) = 1$ implies that $f$ is nonzero. In particular, $f$ maps $\beta'$ to $F$, we know $f$ is a linear functional. i.e., $f$ is what we desire.
	\end{proof}
\end{Exercise}