% === Exercise 2.6.1 ===
\begin{Exercise}
\begin{enumerate}[(a)]
\item[(a)]
\begin{answer}
False.
\end{answer}
\begin{solution}
The codomain of linear functional must be $F$. It should be that every linear functional is a linear transformation.
\end{solution}

\item[(b)]
\begin{answer}
True.
\end{answer}
\begin{solution}
The linear functional maps $F$ into $F$, so we can use a $1\times 1$ matrix to represent.
\end{solution}

\item[(c)]
\begin{answer}
True.
\end{answer}
\begin{solution}
Since $dim(V) = dim(V^*)$.
\end{solution}

\item[(d)]
\begin{answer}
False.
\end{answer}
\begin{solution}
Here we interpret "is" as "equals". Since the codomain of a dual space must be $F$.
\end{solution}

\item[(e)]
\begin{answer}
False.
\end{answer}
\begin{solution}
Let $\beta := \{x_1,x_2,\cdots,x_n\}$ and $\beta^* := \{f_1,f_2,\cdots,f_n\}$. Pick $T(x_i) = 2f_i$ for each $i$, then we observe $T(\beta) = 2\beta^*$.
\end{solution}

\item[(f)]
\begin{answer}
True.
\end{answer}
\begin{solution}
Since $T^t:W^*\to V^*$, then $(T^t)^t:V^{**}\to W^{**}$. So the domain of $(T^t)^t$ is $V^{**}$.
\end{solution}

\item[(g)]
\begin{answer}
True.
\end{answer}
\begin{solution}
Since $dim(V)=dim(V^*)$ and $dim(W)=dim(W^*)$. By hypothesis that $dim(V)=dim(W)$, we have $dim(V^*)=dim(W^*)$ immediately.
\end{solution}

\item[(h)]
\begin{answer}
False.
\end{answer}
\begin{solution}
The codomain of linear functional must be $F$. However, the domain of the derivative of a function might not always be $F$.
\end{solution}

\end{enumerate}
\end{Exercise}