% === Exercise 2.2.6 ===
\begin{Exercise}
\begin{proof}
It suffices to prove $\mathcal{L}(V,W)$ is compatible with the definition from $(VS\ 1)$ to $(VS\ 8)$. We denoted the zero vector in $W$ is $0_W$. Notice that we have known $V$, $W$ are compatible with the definitions since they are vector spaces by hypothesis.

Suppose an arbitrary vector $v$ in $V$. $T_0$ is a zero transformation that $T_0(v) = 0_W$. Here we go by verifying patiently.

\begin{itemize}
\item $\mathbf{(VS\ 1)}$
For all $S,T \in \mathcal{L}(V,W)$, we have
$$
(S+T)(v)
= S(v) + T(v)
= T(v) + S(v)
= (T+S)(v).
$$

\item $\mathbf{(VS\ 2)}$
For all $S,T,U\in \mathcal{L}(V,W)$, we have
\begin{gather*}
((S+T)+U)(v)
= (S+T)(v) + U(v)
= S(v) + T(v) + U(v) \\
= S(v) + (T+U)(v) 
= (S+(T+U))(v).
\end{gather*}

\item $\mathbf{(VS\ 3)}$
For all $T\in \mathcal{L}(V,W)$, we have
$$
(T+T_0)(v)
= T(v) + T_0(v)
= T(v) + 0_W
= T(v).
$$

\item $\mathbf{(VS\ 4)}$
For all $T\in \mathcal{L}(V,W)$, we pick $-T$ such that $(-T)(v) = -T(v)$. Then we have
$$
(T+(-T))(v)
= T(v) + (-T)(v)
= T(v) - T(v)
= 0_W.
$$

\item $\mathbf{(VS\ 5)}$
For all $T\in \mathcal{L}(V,W)$, we have
$$
(1\cdot T)(v)
= 1\cdot T(v)
= T(v).
$$

\item $\mathbf{(VS\ 6)}$
For all $a,b\in F$ and $T\in \mathcal{L}(V,W)$, we have
\begin{align*}
((ab)T)(v)
&= (ab)T(v) \\
&= a(b T(v)) \\
&= a(b T)(v)
\end{align*}

\item $\mathbf{(VS\ 7)}$
For all $a\in F$ and $S,T\in \mathcal{L}(V,W)$, we have
\begin{align*}
(a(S+T))(v)
&= a((S+T))(v)) \\
&= a(S(v) + T(v)) \\
&= a S(v) + a T(v) \\
&= (a S+a T)(v).
\end{align*}

\item $\mathbf{(VS\ 8)}$
For all $a,b\in F$ and $T\in \mathcal{L}(V,W)$, we have
\begin{align*}
((a+b)T)(v)
&= (a+b)T(v) \\
&= (a T)(v) + (b T)(v) \\
&= (a T + b T)(v).
\end{align*}
\end{itemize}
We conclude $\mathcal{L}(V,W)$ is a vector space over $F$ as promised.
\end{proof}
\end{Exercise}