% === Exercise 5.4.19 ===
\begin{Exercise}
	\begin{proof}
		For $k = 1$, $\det(A-t I) = -a_0 - t = (-1)^1(a_0 + t)$, it follows the statement holds.
		
		Suppose the statement holds for $k$. Then for $k+1$, we have
		\begin{align*}
		\det(A-t I) 
		&= \det\begin{pmatrix}
		-t & 0 & \cdots & 0 & -a_0 \\
		1 & -t & \cdots & 0 & -a_1 \\
		0 & 1 & \cdots & 0 & -a_2 \\
		\vdots & \vdots & \ddots & \vdots & \vdots \\
		0 & 0 & \cdots & 0 & -a_{k-1} \\
		0 & 0 & \cdots & 1 & -a_k
		\end{pmatrix} \\
		&= (-t)\det\begin{pmatrix}
		-t & \cdots & 0 & -a_1 \\
		1 & \cdots & 0 & -a_2 \\
		\vdots & \ddots & \vdots & \vdots \\
		0 & \cdots & 1 & -a_k
		\end{pmatrix} + (-1)^{k+2} (-a_0) \det\begin{pmatrix}
		1 & -t & \cdots & 0 & -a_1 \\
		0 & 1 & \cdots & 0 & -a_2 \\
		\vdots & \vdots & \ddots & \vdots & \vdots \\
		0 & 0 & \cdots & 1 & -a_k
		\end{pmatrix} \\
		&= (-t)\left[(-1)^k(a_1+a_2 t+\cdots+a_{k}t^{k}+t^{k+1})\right] + (-1)^{k+2} (-a_0) (1) \\
		&= (-1)^{k+1}(a_1 t+\cdots+a_k t^k+t^{k+1}) + (-1)^{k+1} (a_0) \\
		&= (-1)^{k+1}(a_0 + a_1 t+\cdots+a_k t^k+t^{k+1}).
		\end{align*}
		Hence the statement also holds for $k+1$. 
		
		By induction, we conclude the statement holds for $k\in\mathbb{N}$.
	\end{proof}
\end{Exercise}