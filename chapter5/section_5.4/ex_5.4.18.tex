% === Exercise 5.4.18 ===
\begin{Exercise}
\begin{enumerate}[(a)]
\item
\begin{proof}
Consider
\begin{equation}\label{eq:ex_5.4.18a}
f(t) = (-1)^n t^n + a_{n-1} t^{n-1} + \cdots + a_1 t + a_0 = \det(A-t I).
\end{equation}
We know $f(0) = a_0 = \det(A)$. Hence
$$
A\text{ is invertible} \iff \det(A)\neq 0 \iff a_0 \neq 0.
$$
\end{proof}

\item
\begin{proof}
Consider formula \eqref{eq:ex_5.4.18a}, we have
$$
f(A) = (-1)^n A^n + a_{n-1} A^{n-1} + \cdots + a_1 A + a_0 = \det(O).
$$
Since $\det(O) = 0$, then $(-1)^n A^n + a_{n-1} A^{n-1} + \cdots + a_1 A + a_0 = 0$. Notice that $A$ is invertible and hence from part (a), $a_0\neq 0$. Then divide $-a_0$ on both sides and rearrange it to obtain
$$
\frac{-1}{a_0}\left[ (-1)^n A^n + a_{n-1} A^{n-1} + \cdots + a_1 A \right] = 1.
$$
Now multiply $A^{-1}$ on both sides to conclude
$$
A^{-1} = \frac{-1}{a_0}\left[ (-1)^n A^{n-1} + a_{n-1} A^{n-2} + \cdots + a_1 I_n \right].
$$
\end{proof}

\item
\begin{proof}
Compute $\det(A-t I) = -t^3+2 t^2+t-2$ and $A^2 = \begin{pmatrix}
1 & 6 & 6 \\
0 & 4 & 3 \\
0 & 0 & 1
\end{pmatrix}$.
From part (b), we know
\begin{align*}
A^{-1} 
&= \frac{-1}{a_0}\left[ (-1)A^2 + 2A + I \right] \\
&= \frac{-1}{-2} \left[ (-1)\begin{pmatrix}
1 & 6 & 6 \\
0 & 4 & 3 \\
0 & 0 & 1
\end{pmatrix} + 2\begin{pmatrix}
1 & 2 & 1 \\
0 & 2 & 3 \\
0 & 0 & -1
\end{pmatrix} + \begin{pmatrix}
1 & 0 & 0 \\
0 & 1 & 0 \\
0 & 0 & 1
\end{pmatrix} \right] \\
&= \frac{1}{2}\begin{pmatrix}
2 & -2 & -4 \\
0 & 1 & 3 \\
0 & 0 & -2
\end{pmatrix}.
\end{align*}
\end{proof}
\end{enumerate}
\end{Exercise}