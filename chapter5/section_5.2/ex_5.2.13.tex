% === Exercise 5.2.13 ===
\begin{Exercise}
	\begin{enumerate}[(a)]
		\item
		\begin{proof}
			Let $A = \begin{pmatrix}
			1 & 1 \\
			0 & 2
			\end{pmatrix}$. Then $\lambda = 2$ is an eigenvalue of $A$. We know $\left\{\begin{pmatrix}
			1 \\
			1
			\end{pmatrix} \right\}$ is a basis for $E_{\lambda}$. Consdier $A^t$, $\lambda' = 2$ is also an eigenvalue of $A$. We know $\left\{\begin{pmatrix}
			0 \\
			1
			\end{pmatrix} \right\}$ is a basis for $E_{\lambda'}$.
			
			However $\spann\{(1,1)\} \neq \spann\{(0,1)\}$ since $\spann\{(1,1)\} = \mathbb{R}^2$, $\spann\{(0,1)\} = \mathbb{R}^1$.
		\end{proof}
		
		\item
		\begin{proof}
			Pick an arbitrary eigenvalue $\lambda$. By the Dimension Theorem, we have
			\begin{align*}
			\dim(E_{\lambda}) &= \dim(N(A-\lambda I)) = n-\rank(A-\lambda I); \\
			\dim(E'_{\lambda}) &= \dim(N(A^t-\lambda I)) = n-\rank(A^t-\lambda I).
			\end{align*}
			Since $(A-\lambda I)^t = A^t - \lambda I$, then 
			$$
			\rank(A-\lambda I) = \rank((A-\lambda I)^t) = \rank(A^t-\lambda I).
			$$
			Hence we know for a particular eigenvalue $\lambda$,
			$$
			\dim(E_{\lambda}) = \dim(E'_{\lambda}).
			$$
			
			Since $\lambda$ was arbitrary, the statement follows.
		\end{proof}
		
		\item
		\begin{proof}
			Since $A$ is diagonalizable, then $\det(A-\lambda I)$ can split and $\dim(E_{\lambda})$ equals to the multiplicity of $\lambda$ for each $\lambda$.
			
			Because $\det(A-\lambda I) = \det((A-\lambda I)^t) = \det(A^t-\lambda I)$, $A^t$ can split, too. From part (b), we know $\dim(E_{\lambda}) = \dim(E'_{\lambda})$; hence $\dim(E'_{\lambda})$ equals to the multiplicity of $\lambda$ for each $\lambda$.
			
			They follows $A^t$ is diagonalizable.
		\end{proof}
	\end{enumerate}
\end{Exercise}