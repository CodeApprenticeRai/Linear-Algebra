% === Exercise 5.2.12 ===
\begin{Exercise}
\begin{enumerate}[(a)]
\item
\begin{proof}
We denoted the eigenspace of $T$ corresponding to $\lambda$ by $E_{\lambda,T}$.

Let $x\in E_{\lambda, T}$, then $T(x) = \lambda x$. Since $T$ is invertible, we have $x = \lambda T^{-1} x$. Moreover, by Exercise 5.1.8(a), $0$ is not an eigenvalue of $T$. Hence we can divide $\lambda$ on both sides to obtain $\lambda^{-1} x = T^{-1}(x)$. This means $x\in E_{\lambda^{-1},T^{-1}}$. Conversely, a similar argument establishes $x\in E_{\lambda^{-1},T^{-1}}$ implies $x\in E_{\lambda, T}$.

Since $x$ was arbitrary, we conclude $ E_{\lambda, T} = E_{\lambda^{-1},T^{-1}}$.
\end{proof}

\item
\begin{proof}
Since $T$ is diagonalizable, then $0$ is not an eigenvalue of $T$. Hence there exists $Q$ which is invertible such that $D = Q^{-1} T Q$ where $D$ is a diagonal matrix; also, $D^{-1}$ exists. It follows that $T = Q D Q^{-1}$. This implies that $T^{-1} = (Q D Q^{-1})^{-1} = Q D^{-1} Q^{-1}$. This means $T^{-1}$ is diagonalizable.
\end{proof}
\end{enumerate}
\end{Exercise}