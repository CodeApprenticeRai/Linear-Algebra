% === Exercise 5.2.1 ===
\begin{Exercise}
	\begin{enumerate}[(a)]
		\item[(a)]
		\begin{answer}
			False.
		\end{answer}
		\begin{solution}
			Since $\begin{pmatrix}
			1 & 0 & 0 \\
			0 & 0 & 0 \\
			0 & 0 & 0
			\end{pmatrix}$ is diagonalizable. However, it has two distinct eigenvalues $0$ and $1$.
		\end{solution}
		
		\item[(b)]
		\begin{answer}
			False.
		\end{answer}
		\begin{solution}
			Consider the matrix $\begin{pmatrix}
			4 & 0 & 1 \\
			2 & 3 & 2 \\
			1 & 0 & 4
			\end{pmatrix}$. We know $\begin{pmatrix}
			0 \\
			1 \\
			0
			\end{pmatrix}$, $\begin{pmatrix}
			-1 \\
			0 \\
			1
			\end{pmatrix}$ are eigenvectors corresponding to the eigenvalue $3$; however, they are linearly independent.
		\end{solution}
		
		\item[(c)]
		\begin{answer}
			False.
		\end{answer}
		\begin{solution}
			The zero vector can not be an eigenvector.
		\end{solution}
		
		\item[(d)]
		\begin{answer}
			True.
		\end{answer}
		\begin{solution}
			Let $x\in E_{\lambda_1}\cap E_{\lambda_2}$, then $T x = \lambda_1 x = \lambda_2 x$. It follows that $x=0$ since $\lambda_1 \neq \lambda_2$.
		\end{solution}
		
		\item[(e)]
		\begin{answer}
			True.
		\end{answer}
		\begin{solution}
			It follows from Theorem 5.1.
		\end{solution}
		
		\item[(f)]
		\begin{answer}
			False.
		\end{answer}
		\begin{solution}
			It doesn't satisfy Theorem 5.9. Besides we also need the characteristic polynomial of $T$ can split.
		\end{solution}
		
		\item[(g)]
		\begin{answer}
			True.
		\end{answer}
		\begin{solution}
			Since the vector space is nonzero, then its characteristic polynomial must have degree more than $0$. Hence if it can diagonalize, then it must have an eigenvalue.
		\end{solution}
		
		\item[(h)]
		\begin{answer}
			True.
		\end{answer}
		\begin{solution}
			Since $W_i\cap \sum_{i\neq j} W_j = \{0\}$ by definition. We also know $W_j \in \sum_{k} W_k$. It follows that $W_i \cap W_j =\{0\}$ where $i\neq j$.
		\end{solution}
		
		\item[(i)]
		\begin{answer}
			False.
		\end{answer}
		\begin{solution}
			Consider $W_1 = \spann\{(0,1)\}$, $W_2 = \spann\{1,1\}$, $W_3 = \spann\{(1,0)\}$, then we have $W_i\cap W_j = \{0\}$ for $i\neq j$. However $W_1\cap (W_2+W_3) \neq \{0\}$.
		\end{solution}
		
	\end{enumerate}
\end{Exercise}