% === Exercise 5.1.8 ===
\begin{Exercise}
\begin{enumerate}[(a)]
\item
\begin{proof}
$(\Longrightarrow)$
Suppose to contrary that $\lambda = 0$ is an eigenvalue of $T$. Since $T$ is invertible, then $\det(T) \neq 0$. Consider  $\det(T-\lambda I) = 0$. It follows that $\det(T- 0 I = \det(T) = 0$ which contradicts the supposition. Hence $0$ is not an eigenvalue of $T$.

\vspace{2ex}

$(\Longleftarrow)$
Suppose $T$ is $\mathbf{not}$ invertible, then $\det(T) = 0$. Then if $\lambda = 0$ is an eigenvalue of $T$, it follows that $\det(T-\lambda I) = \det(T) = 0$ holds. We have proven the contraposition.
\end{proof}

\item
\begin{proof}
Let $v$ be some eigenvectors corresponding to $\lambda$. Notice that $T$ is invertible. Consider
\begin{alignat*}{7}
\quad&& T v &= \lambda v \\
\iff&& v &= T^{-1} \lambda v \\
\iff&& \lambda^{-1} v &= T^{-1} v.
\end{alignat*}
This means $\lambda^{-1}$ is an eigenvalue of $T^{-1}$.
\end{proof}

\item
\begin{proof}
We restate part(a) as a matrix $A$ is singular if and only if $0$ is not an eigenvalue of $L_A$. Here is the proof:
$$
A\text{ is singular}
\iff L_A\text{ is not invertible}
\iff 0\text{ is an eigenvalue of } L_A.
$$

We restate part(b) as if $A$ is invertible, then $\lambda$ is an eigenvalue of $L_A$ if and only if $\lambda^{-1}$ is an eigenvalue of $(L_A)^{-1}$. Here is the proof:

Since $A$ is invertible, then $L_A$ is invertible. from part (b), we obtain the statement holds.
\end{proof}
\end{enumerate}
\end{Exercise}