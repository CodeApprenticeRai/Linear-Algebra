% === Exercise 5.1.3 ===
\begin{Exercise}
\begin{enumerate}[(a)]
\item
\begin{answer}
\begin{enumerate}[(i)]
\item $\lambda_1 = 4, \lambda_2 = -1$

\item $v_1 = t\begin{pmatrix}
2 \\
3
\end{pmatrix}, t\neq 0$; $v_2=t\begin{pmatrix}
1 \\
-1
\end{pmatrix}, t\neq 0$.

\item $\begin{pmatrix}
2 & 1 \\
3 & -1
\end{pmatrix}$.

\item $Q = \begin{pmatrix}
2 & 1 \\
3 & -1
\end{pmatrix}$, $D = \begin{pmatrix}
4 & 0 \\
0 & -1
\end{pmatrix}$.
\end{enumerate}
\end{answer}
\begin{solution}
We want to know all eigenvalues of $A$, we need to solve $\det(A-\lambda I) = 0$. This implies
$$
\begin{vmatrix}
1-\lambda & 2 \\
3 & 2-\lambda
\end{vmatrix} =  \lambda^2 -3\lambda - 4
= (\lambda-4)(\lambda+1)
= 0.
$$
Hence $\lambda_1 = 4$, $\lambda_2 = -1$.

Then we want to know all eigenvectors $v_1$ corresponding to $\lambda_1$. Let
$$
B_1 = A-\lambda_1 I = \begin{pmatrix}
-3 & 2 \\
3 & -2
\end{pmatrix}.
$$
Since $x\neq 0$ and $x\in N(L_{B_1})$, we have $B_1 x = 0$. Then we obtain $x \in \left\{t\begin{pmatrix}
2 \\
3
\end{pmatrix}:t\in\mathbb{R}\right\}$. Hence
$$
v_1 = t\begin{pmatrix}
2 \\
3
\end{pmatrix}, t\neq 0.
$$

It turn to know eigenvectors $v_2$ corresponding to $\lambda_2$. Let
$$
B_2 = A-\lambda_2 I = \begin{pmatrix}
2 & 2 \\
3 & 3
\end{pmatrix}.
$$ 
Since $x\neq 0$ and $x\in N(L_{B_2})$, we have $B_2 x = 0$. Then we obtain $x \in \left\{t\begin{pmatrix}
1 \\
-1
\end{pmatrix}:t\in\mathbb{R}\right\}$. Hence
$$
v_2 = t\begin{pmatrix}
1 \\
-1
\end{pmatrix}, t\neq 0.
$$

They follow that $\left\{\begin{pmatrix}
2 \\
3
\end{pmatrix},\begin{pmatrix}
1 \\
-1
\end{pmatrix}\right\}
$ is a basis for $\mathbb{R}^2$ consisting of eigenvectors of $A$. Hence $A$ is diagonalizable. We just pick $Q = \begin{pmatrix}
2 & 1 \\
3 & -1
\end{pmatrix}$, then we have
$$
Q^{-1} A Q = \frac{1}{5}\begin{pmatrix}
1 & 1 \\
3 & -2
\end{pmatrix}\begin{pmatrix}
1 & 2 \\
3 & 2
\end{pmatrix}\begin{pmatrix}
2 & 1 \\
3 & -1
\end{pmatrix} = \begin{pmatrix}
4 & 0 \\
0 & -1
\end{pmatrix} =: D.
$$
Finally, we complete the computation.
\end{solution}
\end{enumerate}
\end{Exercise}